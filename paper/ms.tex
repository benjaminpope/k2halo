%%
%% Beginning of file 'sample62.tex'
%%
%% Modified 2018 January
%%
%% This is a sample manuscript marked up using the
%% AASTeX v6.2 LaTeX 2e macros.
%%
%% AASTeX is now based on Alexey Vikhlinin's emulateapj.cls 
%% (Copyright 2000-2015).  See the classfile for details.

%% AASTeX requires revtex4-1.cls (http://publish.aps.org/revtex4/) and
%% other external packages (latexsym, graphicx, amssymb, longtable, and epsf).
%% All of these external packages should already be present in the modern TeX 
%% distributions.  If not they can also be obtained at www.ctan.org.

%% The first piece of markup in an AASTeX v6.x document is the \documentclass
%% command. LaTeX will ignore any data that comes before this command. The 
%% documentclass can take an optional argument to modify the output style.
%% The command below calls the preprint style  which will produce a tightly 
%% typeset, one-column, single-spaced document.  It is the default and thus
%% does not need to be explicitly stated.
%%
%%
%% using aastex version 6.2
\documentclass[modern]{aastex62}

%% The default is a single spaced, 10 point font, single spaced article.
%% There are 5 other style options available via an optional argument. They
%% can be envoked like this:
%%
%% \documentclass[argument]{aastex62}
%% 
%% where the layout options are:
%%
%%  twocolumn   : two text columns, 10 point font, single spaced article.
%%                This is the most compact and represent the final published
%%                derived PDF copy of the accepted manuscript from the publisher
%%  manuscript  : one text column, 12 point font, double spaced article.
%%  preprint    : one text column, 12 point font, single spaced article.  
%%  preprint2   : two text columns, 12 point font, single spaced article.
%%  modern      : a stylish, single text column, 12 point font, article with
%% 		  wider left and right margins. This uses the Daniel
%% 		  Foreman-Mackey and David Hogg design.
%%  RNAAS       : Preferred style for Research Notes which are by design 
%%                lacking an abstract and brief. DO NOT use \begin{abstract}
%%                and \end{abstract} with this style.
%%
%% Note that you can submit to the AAS Journals in any of these 6 styles.
%%
%% There are other optional arguments one can envoke to allow other stylistic
%% actions. The available options are:
%%
%%  astrosymb    : Loads Astrosymb font and define \astrocommands. 
%%  tighten      : Makes baselineskip slightly smaller, only works with 
%%                 the twocolumn substyle.
%%  times        : uses times font instead of the default
%%  linenumbers  : turn on lineno package.
%%  trackchanges : required to see the revision mark up and print its output
%%  longauthor   : Do not use the more compressed footnote style (default) for 
%%                 the author/collaboration/affiliations. Instead print all
%%                 affiliation information after each name. Creates a much
%%                 long author list but may be desirable for short author papers
%%
%% these can be used in any combination, e.g.
%%
%% \documentclass[twocolumn,linenumbers,trackchanges]{aastex62}
%%
%% AASTeX v6.* now includes \hyperref support. While we have built in specific
%% defaults into the classfile you can manually override them with the
%% \hypersetup command. For example,
%%
%%\hypersetup{linkcolor=red,citecolor=green,filecolor=cyan,urlcolor=magenta}
%%
%% will change the color of the internal links to red, the links to the
%% bibliography to green, the file links to cyan, and the external links to
%% magenta. Additional information on \hyperref options can be found here:
%% https://www.tug.org/applications/hyperref/manual.html#x1-40003
%%
%% If you want to create your own macros, you can do so
%% using \newcommand. Your macros should appear before
%% the \begin{document} command.
%%
% added by DH
\usepackage{xspace}
\usepackage{amsmath}
\usepackage{newunicodechar,graphicx}
\usepackage{longtable}


\DeclareRobustCommand{\okina}{%
  \raisebox{\dimexpr\fontcharht\font`A-\height}{%
    \scalebox{0.8}{`}%
  }%
}
\newunicodechar{ʻ}{\okina}

\newcommand{\numax}{\mbox{$\nu_{\rm max}$}\xspace}
\newcommand{\Dnu}{\mbox{$\Delta \nu$}\xspace}
\newcommand{\dnu}{\mbox{$\delta \nu$}\xspace}
\newcommand{\muHz}{\mbox{$\mu$Hz}\xspace}
\newcommand{\teff}{\mbox{$T_{\rm eff}$}\xspace}
\newcommand{\logg}{\mbox{$\log g$}\xspace}
\newcommand{\feh}{\mbox{$\rm{[Fe/H]}$}\xspace}
\newcommand{\msun}{\mbox{$\mathrm{M}_{\sun}$}\xspace}
\newcommand{\rsun}{\mbox{$\mathrm{R}_{\sun}$}\xspace}
\newcommand{\hipparcos}{\textit{Hipparcos}\xspace}
\newcommand{\gaia}{\textit{Gaia}\xspace}


\newcommand{\vdag}{(v)^\dagger}
\newcommand\aastex{AAS\TeX}
\newcommand\latex{La\TeX}
\newcommand\kepler{\emph{Kepler}\,}
\newcommand\ktwo{\emph{K2}\,}

\definecolor{linkcolor}{rgb}{0.1216,0.4667,0.7059}


\usepackage{xcolor, fontawesome}
\definecolor{twitterblue}{RGB}{64,153,255}
\newcommand\twitter[1]{\href{https://twitter.com/#1 }{\textcolor{twitterblue}{\faTwitter}\,\tt \textcolor{twitterblue}{@#1}}}
% \usepackage{ccaption}
% \usepackage{graphicx}	% Including figure files
% \usepackage{amsmath}	% Advanced maths commands
% \usepackage{amssymb}	% Extra maths symbols


%% Tells LaTeX to search for image files in the 
%% current directory as well as in the figures/ folder.
\graphicspath{{./}{figures/}}

%% Reintroduced the \received and \accepted commands from AASTeX v5.2
\received{January 1, 2019}
\revised{January 7, 2019}
\accepted{\today}
%% Command to document which AAS Journal the manuscript was submitted to.
%% Adds "Submitted to " the arguement.
\submitjournal{ApJ}

%% Mark up commands to limit the number of authors on the front page.
%% Note that in AASTeX v6.2 a \collaboration call (see below) counts as
%% an author in this case.
%
%\AuthorCollaborationLimit=3
%
%% Will only show Schwarz, Muench and "the AAS Journals Data Scientist 
%% collaboration" on the front page of this example manuscript.
%%
%% Note that all of the author will be shown in the published article.
%% This feature is meant to be used prior to acceptance to make the
%% front end of a long author article more manageable. Please do not use
%% this functionality for manuscripts with less than 20 authors. Conversely,
%% please do use this when the number of authors exceeds 40.
%%
%% Use \allauthors at the manuscript end to show the full author list.
%% This command should only be used with \AuthorCollaborationLimit is used.

%% The following command can be used to set the latex table counters.  It
%% is needed in this document because it uses a mix of latex tabular and
%% AASTeX deluxetables.  In general it should not be needed.
%\setcounter{table}{1}

%%%%%%%%%%%%%%%%%%%%%%%%%%%%%%%%%%%%%%%%%%%%%%%%%%%%%%%%%%%%%%%%%%%%%%%%%%%%%%%%
%%
%% The following section outlines numerous optional output that
%% can be displayed in the front matter or as running meta-data.
%%
%% If you wish, you may supply running head information, although
%% this information may be modified by the editorial offices.
\shorttitle{The K2 Bright Star Survey}
\shortauthors{B. J. S. Pope et al.}
%%
%% You can add a light gray and diagonal water-mark to the first page 
%% with this command:
% \watermark{text}
%% where "text", e.g. DRAFT, is the text to appear.  If the text is 
%% long you can control the water-mark size with:
%  \setwatermarkfontsize{dimension}
%% where dimension is any recognized LaTeX dimension, e.g. pt, in, etc.
%%
%%%%%%%%%%%%%%%%%%%%%%%%%%%%%%%%%%%%%%%%%%%%%%%%%%%%%%%%%%%%%%%%%%%%%%%%%%%%%%%%

%% This is the end of the preamble.  Indicate the beginning of the
%% manuscript itself with \begin{document}.

\begin{document}

\title{The K2 Bright Star Survey}

\correspondingauthor{Benjamin J. S. Pope \twitter{fringetracker}}
\email{benjamin.pope@nyu.edu}

\author[0000-0003-2595-9114]{Benjamin J. S. Pope}
\affiliation{Center for Cosmology and Particle Physics, Department of Physics, New York University, 726 Broadway, New York, NY 10003, USA}
\affiliation{Center for Data Science, New York University, 60 Fifth Ave, New York, NY 10011, USA}
\affiliation{NASA Sagan Fellow}

\author[0000-0002-6980-3392]{Timothy R. White}
\affiliation{Research School of Astronomy and Astrophysics, Mount Stromlo Observatory, The Australian National University, Canberra, ACT 2611, Australia}

\author[0000-0003-1822-7126]{Conny Aerts}
\affiliation{Instituut voor Sterrenkunde, KU Leuven, Celestijnenlaan 200D, B-3001 Leuven, Belgium}
\affiliation{Department of Astrophysics, IMAPP, Radboud University Nijmegen, P.O. Box 9010, NL-6500 GL Nijmegen, The Netherlands}

\author[0000-0003-1453-0574]{Suzanne Aigrain}
\affiliation{Oxford Astrophysics, Denys Wilkinson Building, University of Oxford, OX1 3RH, Oxford, UK}

\author[0000-0001-5222-4661]{Timothy R. Bedding}
\affiliation{Sydney Institute for Astronomy, School of Physics A28, The University of Sydney, NSW 2006, Australia}
\affiliation{Stellar Astrophysics Centre, Department of Physics and Astronomy, Aarhus University, DK-8000 Aarhus C, Denmark}

\author[0000-0001-9879-9313]{Tabetha Boyajian}
\affiliation{Department of Physics and Astronomy, Louisiana State University, 202 Nicholsom Hall, Baton Rouge, LA 70803, USA}

\author[0000-0003-1853-6631]{Orlagh L. Creevey}
\affiliation{Universit\'{e} C\^{o}te d'Azur, Observatoire de la C\^{o}te d'Azur, CNRS, Laboratoire Lagrange, Bd de l'Observatoire, CS 34229, 06304 Nice cedex 4, France}

\author[0000-0003-1540-8562]{Will M. Farr}
\affiliation{Center for Computational Astrophysics, Flatiron Institute, 162 Fifth Ave, New York, NY 10010, USA}
\affiliation{Department of Physics and Astronomy, Stony Brook University, Stony Brook, NY 11794, USA}

\author[0000-0003-2866-9403]{David W. Hogg}
\affiliation{Center for Cosmology and Particle Physics, Department of Physics, New York University, 726 Broadway, New York, NY 10003, USA}
\affiliation{Center for Data Science, New York University, 60 Fifth Ave, New York, NY 10011, USA}
\affiliation{Max-Planck-Institut f\"{u}r Astronomie, K\"{o}nigstuhl 17, D-69117 Heidelberg}
\affiliation{Flatiron Institute, 162 Fifth Ave, New York, NY 10010, USA}

\author[0000-0001-8832-4488 ]{Daniel Huber}
\affiliation{Institute for Astronomy, University of Hawai\okina i, 2680 Woodlawn Drive, Honolulu, HI 96822, USA}
\affiliation{SETI Institute, 189 Bernardo Avenue, Mountain View, CA 94043, USA}
\affiliation{Stellar Astrophysics Centre, Department of Physics and Astronomy, Aarhus University, DK-8000 Aarhus C, Denmark}


\author{friends}
%% Note that the \and command from previous versions of AASTeX is now
%% depreciated in this version as it is no longer necessary. AASTeX 
%% automatically takes care of all commas and "and"s between authors names.

%% AASTeX 6.2 has the new \collaboration and \nocollaboration commands to
%% provide the collaboration status of a group of authors. These commands 
%% can be used either before or after the list of corresponding authors. The
%% argument for \collaboration is the collaboration identifier. Authors are
%% encouraged to surround collaboration identifiers with ()s. The 
%% \nocollaboration command takes no argument and exists to indicate that
%% the nearby authors are not part of surrounding collaborations.

%% Mark off the abstract in the ``abstract'' environment. 
\begin{abstract}
While the \kepler\ Mission was designed to look at tens of thousands of faint stars ($V \gtrsim 12$), brighter stars, which saturate the detector, are important because they can be and have been observed by other instruments at very high signal-to-noise ratio. By considering the unsaturated scattered light `halo' around these stars we can and do retrieve precise light curves of most of the brightest stars in \ktwo fields from Campaign~4 onwards. The halo method is highly agnostic about the cause and form of systematics and we show that it nevertheless it is effective at extracting light curves from both normal and saturated stars. The key methodology is to optimize the weights of a linear combination of pixel time series with respect to an objective function. We test a range of such objective functions, finding that generalizations of Total Variation perform well on both saturated and unsaturated \ktwo targets. Applying this to the bright stars across the \ktwo Campaigns, this reveals stellar variability ubiquitously, including effects of stellar pulsation, rotation, and binarity. Here we describe our pipeline, and present a catalogue of the bright stars studied, with classifications and parametrizations of their variability and remarks on interesting objects. These light curves are publicly available as a High Level Science Product from the Mikulski Archive for Space Telescopes (MAST). \href{https://github.com/benjaminpope/k2halo}{\color{linkcolor}\faGithub} % git link shamelessly borrowed from Luger!
\end{abstract}

%% Keywords should appear after the \end{abstract} command. 
%% See the online documentation for the full list of available subject
%% keywords and the rules for their use.
% \keywords{editorials, notices --- 
% miscellaneous --- catalogs --- surveys}


\section{Introduction} 
\label{sec:intro}

% intro kepler
The \kepler Space Telescope was launched with a main goal of determining the frequency of Earth-sized planets around Solar-like stars \citep{2010sci...327..977b}, a goal which it has substantially achieved \citep[e.g.][]{2013ApJ...766...81F,2013PNAS..11019273P,2014ApJ...795...64F}. In order to explore these populations it was necessary to observe hundreds of thousands of stars, with the consequence that the \kepler the exposure time and gain were set to optimally observe eleventh or twelfth-magnitude stars, while bright stars are saturated and these saturated stars were intentionally avoided. In the two-wheeled revival as the \ktwo mission, the \kepler telescope observed a sequence of ecliptic-plane fields containing many more very-saturated stars. While it is difficult to obtain precise light curves of these stars because of their saturation, they are some of the most-valuable targets to follow up with photon-hungry methods such as interferometry or high-resolution spectroscopy, and they typically have long histories of previous observations. 

% describe saturation
The \kepler detector saturates at a magnitude of $K_p \sim 11.3$ in both long- (30\,min) and short (1\,min)-cadence data, as these both represent sums of 6~s exposures \citep{Gilliland2010}. For objects brighter than this, additional photons create excess electrons which `bleed' into adjacent pixels in both directions along the column containing the bright star. Simple aperture photometry (SAP) -- adding all the flux contained in a window around the bleed column -- has recovered light curves with precisions close to the photon noise limit of stars as bright as 16~Cyg~AB, $\theta$~Cyg and RR~Lyr \citep[e.g.][]{Kolenberg2011,2013MNRAS.433.1262W,Guzik2016}. In the nominal \kepler mission this was not attempted for all such bright stars, and in \ktwo, the several-pixel spacecraft motion significantly increased the size of the required apertures for SAP photometry of very saturated stars, while also making their systematics more difficult to deal with. While the second-version pixel-level-decorrelation (PLD) pipeline EVEREST~2.0 was able to correct systematics in saturated SAP photometry \citep{everest2}, this is rendered impossible for the very brightest stars whose bleed columns may run to the edge of the detector. Furthermore, bandwidth constraints meant that pixel data were not downloaded for many bright targets in \ktwo. 

% briefly intro the halo method and halo literature
In order to recover precise photometry of the brightest stars in \ktwo, we have therefore developed two main approaches, `smear' and `halo' photometry. Smear photometry \citep{Pope2016} uses collateral `smear' calibration data to obtain a 1-D spatial profile with $\sim 1/1000$ of the flux on each CCD. This can be processed recover light curves of stars which were not necessarily conventionally targeted and downloaded with active pixels, because smear data are recorded for all columns. The main disadvantage of this method is that it confuses all stars in the same column, which means that in crowded fields smear light curves tend to be significantly contaminated. 

% halo
The more precise method of halo photometry, which is the subject of this paper, uses the broad `halo' of scattered light around a saturated star to recover relative photometry, by constructing a light curve as a linear combination of individual pixel time series and minimizing a Total Variation objective function (TV-min). It has been employed for example on the Pleiades \citep{White2017} and the brightest-ever star on \kepler silicon, Aldebaran \citep{Farr2018}, recovering photometry with a precision nearly that of normally-obtained \ktwo observations of unsaturated stars. Unlike smear, this requires downloading data out to a 12--20~pixel radius around each star, and has accordingly only been possible for stars that were specifically targeted with apertures optimized for this method and for a small number of other stars for which this is fortuitously the case. The pixel requirements for this are sufficiently low that, with the help of the \ktwo Guest Observer office, such apertures were obtained for most of the bright targets from Campaign~4 onwards.

% describe contents of paper: code, catalog, categorization, conclusions
In this Paper we describe numerical experiments testing the TV-min method and extending it to generalizations with different exponents and timescales. We show that the method as previously employed applying standard TV-min is suboptimal, and gain a modest improvement from taking finite differences close to the timescale of \ktwo thruster firings. 
We go on to present complete catalog of long-cadence \ktwo halo light curves, which we have made publicly available. We have employed halo photometry on all stars targeted with appropriate apertures, and have done a preliminary characterization of interesting astrophysical variability. We also document the main changes in the halo data reduction pipeline, \texttt{halophot}, with respect to previous releases. These include oscillating red giants, pulsating and quiet main sequence stars, and eclipsing binaries, many of which are among the brightest objects of their type to have been observed with space photometry. We hope that this diverse catalog of light curves will be useful for an equally-diverse range of astrophysical investigations. 

\section{Halo Photometry Method}
\label{method}


% describe halo algorithm
This method was first described by \citet{White2017} and applied to the Pleiades' Seven Sisters, and was applied to Aldebaran with further developments by \citet{Farr2018}. Because SAP is not possible, we consider instead the unsaturated pixels $p_j$ at the wings of the broad and structured PSF. We construct a light curve as a linear combination of these time series with weights $w_j$, so that flux at cadence $i$ is 

\begin{equation}
    f_i \equiv \sum_j {w_j p_{ij}}.
\end{equation}

\noindent In our updated pipeline presented here, the weights are chosen to minimize an objective function

\begin{equation}
    Q_{k,l} \equiv {\sum_i{|f_i - f_{i-\delta}|^k}},
\end{equation}

\noindent subject to the constraints 

\begin{align}
\forall_j w_j &> 0\\
\sum_{i=1}^{N} f_i  &= N.
\end{align}

This is a classic convex optimization program with constraints, which we solve with the \texttt{scipy} \citep{scipy} L-BFGS-B nonlinear optimization code \citep{lbfgsb}. $Q_{k,l}$ has analytic derivatives with respect to $w_j$ \citep[calculated with \texttt{autograd};][]{autograd}, and it is therefore extremely fast to optimize and converges well on a global solution. In practice, for computational reasons we optimize over parameters $\tilde{w}_j$ such that $w_j = \text{softmax}(\tilde{w}_j$, as this satisfies the constraint that $\forall_j w_j > 0$, and while this also constrains their sum to be unity, we renormalize $f$ to satisfy its normalization constraint before calculating the objective function and this additional constraint is removed again.

The objective function $Q_{k,l}$ is the $L_k$ norm on a `lagged' finite difference with a lag parameter $\delta$. For $k = 1$ and $\delta = 1$, $Q_{1,1}$ is the standard Total Variation objective (TV) used in previous halo papers \citep[e.g.][]{White2017,Farr2018}, and can be seen as the L1 norm on the derivative of $f$ or as a discrete approximation to its arc length. The L2 Variation (L2V) with $k=2$ is sometimes referred to in image processing literature as the `smoothness' regularizer, as it seeks to penalize large gradients without necessarily making them sparse. The lag parameter $\delta$ allows for flexibility in modelling systematics occurring at different timescales from cadence-to-cadence, and we investigate its effects below. The order parameter $k$ allows for flexibility in how sensitive we are to normally-distributed versus long-tailed noise. For convenience in the rest of this paper, we will refer to the $k=1$ case as TV, the $k=2$ case as L2V, and the $k=3$ case as L3V.

% benchmarking with lags and objective functions

In order to choose the values for $k$ and $\delta$ in our objective function, we have selected the quiet star 36~Ophiuchi, the lowest-mass main sequence star in the sample of stars with halo apertures, and one with very little high frequency variability detected or predicted, which was also observed at short cadence. We choose the 6.5~hour Combined Differential Photometric Precision \citep[CDPP,][]{cdpp} as implemented in \texttt{lightkurve} \citep{lightkurve} as a proxy for the `noise' in a lightcurve, with lower being better. We calculate halo lightcurves and their CDPPs for $k \in \{1,2,3\}$, and $\delta \in [1,50]$ for long cadence and for various valies of $\delta \in [1,2500]$ for short cadence data. The results are displayed in Figures~\ref{cdpps_lc} and~\ref{cdpps_sc}. We find that for long cadence data, $k=1$ (TV) and a lag $\delta=10$ provide the best CDPP. This is unsurprising: that this is one cadence shorter than the 12~cadence thruster firing period. In this context we can understand the optimum as suppressing systematics on the same timescale as they occur. On the other hand, for short cadence data, performance at short lags is very poor but the method performs similarly for $k \in \{1,2\}$ with slow improvement with larger $\delta$, and performs very poorly for $k=3$ at all lags. 

% \begin{figure*}
% \plottwo{lag_lc_cdpp.pdf}{lag_sc.pdf}
% \caption{Behaviour of long cadence (left) and short cadence (right) 6.5~hour CDPP for the quiet dwarf 36~Ophiuchi as a function of lag parameter $\delta$. For long cadence this shows a minimum for L1 norm and $\delta=10$, i.e. for objective function $Q_{1,10}$. For short cadence CDPP continuously improves for higher lags and shows no strong differences between L1 and L2~norms, while L3 performs poorly.}
% \label{cdpps}
% \end{figure*}


% \begin{figure*}
% \plottwo{lag_lc_cdpp.pdf}{lag_lc_sys.pdf}
% \caption{Behaviour of long cadence 6.5~hour CDPP (left) and $4 c/d$ systematics power (right) for the quiet dwarf 36~Ophiuchi as a function of lag parameter $\delta$. CDPP shows a minimum for L1 norm and $\delta=10$, i.e. for objective function $Q_{1,10}$.}
% \label{cdpps}
% \end{figure*}


\begin{figure*}
\plotone{lag_opt_both.pdf}
\caption{Behaviour of long cadence 6.5~hour CDPP (bottom) and $4 c/d$ systematics power (top) for the quiet dwarf 36~Ophiuchi as a function of lag parameter $\delta$. CDPP shows a minimum for L1 norm and $\delta=10$, i.e. for objective function $Q_{1,10}$, which is marked with a blue dashed vertical line. This does not correspond to an optimum in systematic power, which is slightly lower for smaller $\delta$. Nevertheless, we have chosen $\delta=10$ for the light curves in this catalog because of its improvement in overall CDPP as a measure of planet detection efficiency and overall light curve quality.}
\label{cdpps_lc}
\end{figure*}

\begin{figure}
\plotone{lag_sc.pdf}
\caption{Behaviour of short cadence 6.5~hour CDPP for the quiet dwarf 36~Ophiuchi as a function of lag parameter $\delta$. CDPP continuously improves for higher lags and shows no strong differences between L1 and L2~norms, while L3 performs poorly.}
\label{cdpps_sc}
\end{figure}


%  improvements
Unlike other methods for calibrating \kepler systematics, other than the value of $\delta$, no knowledge of the spacecraft motion or the behaviour of an ensemble of other stars is used to inform our algorithm. The method is a self-calibration that is independent of the details of the systematics it is calibrating, operating on the assumption that a single signal is present across many individual time series which otherwise are contaminated by noise. It is therefore likely that significant improvements can be made to the method by including cotrending basis vectors with mean zero and whose weights are allowed to be negative, which would represent systematics which are common to all pixels in the halo aperture and therefore masquerade as signal. Any linear combination of convex objective functions is itself convex, and future extensions to the method could enforce combinations of different lags and orders to better represent systematics occurring on different timescales (e.g. thruster firings, red noise) and with different levels of smoothness.

% describe changes with respect to Pleiades and Aldebaran papers: deathstar

In addition to expanding the range of possible objective functions, we have also added a feature `\texttt{deathstar}' to deal with contamination. We apply the watershed-based image segmentation algorithm from \textsc{k2p2} \citep{k2p2} to the input target pixel file datacube to identify possible background sources and cut them out. \textcolor{red}{More here from Tim.} Other than this, we have adopted less-aggressive quality flagging, having found that many cadences were being classified as bad quality for spurious `cosmic ray' events, which were actually caused by a combination of saturation and spacecraft motion. We now iteratively sigma-clip outliers and retain cadences with the \texttt{lightkurve} default quality mask.

% incorporation of k2sc
While the halo procedure produces in most cases a fairly clean light curve, there are nevertheless residual systematic errors related to spacecraft motion. In order to correct these, we employ the \textsc{k2sc} code \citep{Aigrain2015,k2sc}, which simultaneously models a light curve as a 3D Gaussian Process (GP) in time and predicted position (the K2 standard data product \textsc{pos\_corr}) in pixels $(x,y)$. The model prediction in time for fixed position is then a nonparametric model of the stellar variability, and the prediction for the $x,y$ component evaluated for fixed time represents the pointing systematics. We subtract the systmatics model from the input fluxes to obtain a final corrected flux, which is the time series we use and recommend for science. The target pixel files for C91, C92, and C101 include no position information, and there are no halo apertures for C112. As a result \textsc{k2sc}-corrected data are not available for these targets.

% figure 

\begin{figure*}
\plotone{../release/c14/hlsp_halo_k2_llc_200182931_-c14_kepler_v1_lc.pdf}
\caption{Summary plots for \textsc{k2sc}-corrected final halo light curve for $\rho$~Leonis. The top three panels illustrate \textsc{k2sc} systematics correction: at the top, flux minus the GP time trend (blue dots) with GP $x,y$ trend superimposed (orange line); in the middle, flux minus GP $x,y$ components with GP time trend superimposed, and in green, a fifteenth-order polynomial trend; at the bottom the `whitened' light curve with flux minus both GP components. Middle two panels: log-flux map (left) and halo log-weight map (right). Bottom two panels: periodograms in linear (top) and log (bottom) units of the residuals of the corrected light curve minus the long term polynomial trend. Plots of this form are available in supplementary online material for all long-cadence stars, together with similar plots for all short-cadence stars but without \textsc{k2sc}. The period at maximum power (16\,d) is marked on all plots, though for $\rho$~Leonis all variability is consistent with red noise (Bowman et al., 2019).}
\end{figure*}


\section{Sample}
\label{sec:sample}

% link to tables 

The full sample of stars for which halo apertures were obtained is listed in Table~\ref{table_all}. While some very bright stars were observed with conventional apertures as part of these programs, simple aperture photometry is satisfactory on these targets and we exclude them from the present discussion and data release, which is oriented strictly towards targets only observable with halo photometry. 
% normal apertures in 6
We make an exception for Spica, which was observed in Campaign~6 without a halo aperture but in Campaign~17 with a halo aperture. In Campaign~6 it was assigned a normal aperture by mistake and simple aperture photometry performed extremely poorly, so we have processed it with the halo pipeline. The stars in Campaign~18 were also on-silicon in Campaign~5, but were not assigned apertures suitable for halo photometry in C5. A possible further extension of the present work would be to recover C5 light curves for these objects using either or both of smear or modified halo photometry.

% short cadence availability 
Seven stars in Campaign~13 and one in Campaign~16 were not only assigned long-cadence halo apertures, but also downloaded at short cadence. For these targets we have provided both long and short cadence reductions. Following the analysis in Section~\ref{method} showing the insensitivity of short cadence CDPP to lags longer than $\sim 100$~cad and on choice of objective function, for consistency with long cadence we have adopted a 300~cadence lag (i.e. $30 \times$ the long cadence lag of 10) and the L1 TV objective function. With their long timeseries the short cadence stars are computationally intractable for the Gaussian Process model in \textsc{k2sc} and we present otherwise uncalibrated halo lightcurves.

% breakdown of main sequence vs giants

% note interesting targets: Aldebaran; Spica; Hyades; rho Leo

Analyses for several of the objects here have been previously published, but we here provide the first public data releases for the Pleiades' Seven Sisters \citep{White2017}, Aldebaran \citep{Farr2018}, $\iota$~Lib \citep{Buysschaert2018}, and $\epsilon$~Tau \citep{Arentoft2019}, as well as $\rho$~Leo which was studied with halo pixels but without our objective functions \citep{Aerts2018}.

\section{Discussion}
\label{sec:discussion}

% how well did we do?
\textcolor{red}{How well do we do? How do we quantify our lightcurve SNR?}


\subsection{Oscillating Red Giants}
\label{sec:rgs}
31 of the evolved stars in our sample have detectable stochastically-excited solar-like acoustic oscillations. In the asymptotic limit, these consist of a comb of modes separated by the large frequency separation \Dnu, approximately the sound-crossing-time of the star, with a Gaussian envelope centred on the frequency of maximum power \numax, which scales with the acoustic cutoff frequency at the star's surface. These can be used to constrain stellar fundamental parameters, and detailed studies of the deviations from this asymptotic limit for acoustic $p$-modes, for example due to their interaction with gravity-wave $g$-modes, can be used to accurately determine the stellar evolutionary state, for example to distinguish hydrogen shell burning red giant branch from helium core burning red clump stars. 

Using the Sydney pipeline \citep{Huber2009} with modifications to the extraction of \Dnu detailed in \citep{Yu2018}, we extract the global asteroseismic parameters \numax and \Dnu for all 31 giants for which oscillations are detected. These parameters are listed in Table~\ref{rgs}. High precision spectroscopy of these stars would permit detailed stellar modelling and the extraction of precise elemental abundances, which would make these stars valuable as benchmarks for large spectroscopic surveys or testing detailed stellar models. This sample would be a valuable addition to the~36 Gaia FGK benchmark stars \citep{gaiabenchmark1,gaiabenchmark3,2018RNAAS...2c.152J} and the~33 \kepler Smear Campaign spectroscopic benchmark red giants (Pope et al., 2019, under review).

% RGs, Classical pulsators 
\subsection{Classical Variables}
\label{sec:variables}
\textcolor{red}{Variability catalog?}


\subsection{Chemically-Peculiar Stars}

% interesting EB - 98 Tau
The chemically-peculiar A0V star 98~Tau is of special interest for studies of surface inhomogeneity. We detect variability with a fundamental period of 1.74~d with twice as much power at the first harmonic ($P = 0.87$\,d), which is consistent with $\alpha^2$\,CVn spot modulation from a rapidly-rotating star with a period of 1.74\,d. This star also experiences a transit of depth 0.03, which for a 1.87\,\rsun typical A0V star imply an 0.3 \rsun companion, almost certainly of stellar mass. With rotational modulation and an eclipse to break degeneracies, models such as \texttt{starry} \citep{starry} can infer surface brightness maps and reveal the spatial distribution of the star's chemical peculiarity. 

\section{Conclusions}
\label{sec:conclusions}

% pretty good sample

% do hot star asteroseismology and red giants as benchmarks, interferometry
Some of the objects presented here are the subject of more detailed work in preparation, namely Spica (Buzasi et al., in prep.) and the Hyades giants (White et al., in prep.). In addition to this, we aim to separately publish asteroseismic catalogues of the red giants (Yu et al., in prep.) and main-sequence stars (Greklek-McKeon et al., in prep.). 

% do k2 smear after that
The sample of K2 bright stars presented here only includes those with halo apertures, but while some others are available conventionally, many were not assigned target pixels and were not downloaded at all. Smear photometry has been used to recover the brightest otherwise-unobserved stars in nominal \kepler (Pope et al., in prep.), and this can also be done in K2, although in the latter case the sample is much smaller due to competition with halo apertures and the systematics correction is more challenging. A natural extension of both pieces of work would be to produce smear light curves of all bright stars without halo apertures in K2, which would finally make the \kepler extended mission magnitude-complete at the bright end. 

% tess halo 
The halo method naturally extends to other contexts where simple aperture photometry is not possible, such as for saturated stars observed by the Transiting Exoplanet Survey Satellite \citep[TESS;][]{tess}. Although the saturation limit is brighter ($T_{mag} \sim 6$) and this problem accordingly affects fewer stars and less badly, there are situations such as for $\alpha$~Centauri or $\beta$~Hydri where the bleed column reaches the edge of the chip and a SAP light curve is irrecoverable. We expect that TV-min halo photometry will therefore be valuable in ensuring that TESS can observe even the very brightest stars.

\section*{Acknowledgements} % add your acknowledgements text here!

This work was performed in part under contract with the Jet Propulsion Laboratory (JPL) funded by NASA through the Sagan Fellowship Program executed by the NASA Exoplanet Science Institute. TRW acknowledges the support of the Australian Research Council (grant DP150100250) and the Villum Foundation (research grant 10118). The halo apertures were kindly provided by the K2 team as part of the Guest Observer programs GO6081-7081, GO8025, GO9923, GO10025, GO11047-13047, GO14003-16003, and GO17051-19051, and as a Director's Discretionary Time program in Campaign 4 as GO4901. We are grateful for the associated funding provided by the K2 GO office which has been essential in bringing this project to fruition.

This project was developed in part at the Building Early Science with TESS meeting, which took place in March 2019 at the University of Chicago.

BJSP acknowledges being on the traditional territory of the Lenape Nations and recognizes that Manhattan continues to be the home to many Algonkian peoples. We give blessings and thanks to the Lenape people and Lenape Nations in recognition that we are carrying out this work on their indigenous homelands. We would like to acknowledge the Gadigal Clan of the Eora Nation as the traditional owners of the land on which the University of Sydney is built and on which some of this work was carried out, and pay their respects to their knowledge, and to their elders past, present, and future.
%

This research made use of NASA's Astrophysics Data System; the SIMBAD database, operated at CDS, Strasbourg, France. Some of the data presented in this paper were obtained from the Mikulski Archive for Space Telescopes (MAST). STScI is operated by the Association of Universities for Research in Astronomy, Inc., under NASA contract NAS5-26555. Support for MAST for non-HST data is provided by the NASA Office of Space Science via grant NNX13AC07G and by other grants and contracts. We acknowledge the support of the Group of Eight universities and the German Academic Exchange Service through the Go8 Australia-Germany Joint Research Co-operation Scheme. 

\software{IPython  \citep{PER-GRA:2007}; SciPy \citep{scipy}; and Astropy, a community-developed core Python package for Astronomy \citep{astropy}.}

%%%%%%%%%%%%%%%%%%%%%%%%%%%%%%%%%%%%%%%%%%%%%%%%%%

%%%%%%%%%%%%%%%%%%%% REFERENCES %%%%%%%%%%%%%%%%%%


\bibliography{ms}

\newpage
\appendix
% \section*{Sample}

% \begin{table*}
\caption{Stars in Campaigns 7-8 observed with halo photometry in K2.\label{table_0}}
\begin{tabular}{ccccccc}
\hline \hline
Name & EPIC & Spectral & V & Campaign & Notes & Class \\
 &  &  &  &  &  &  \\
\hline
Alcyone & 200007767 & B7III & 2.986 & 4 & \tablenotemark{a} & SPB \\
Atlas & 200007768 &  & 3.763 & 4 & \tablenotemark{a} & SPB \\
Electra & 200007769 & B6IIIe & 3.851 & 4 & \tablenotemark{a} & SPB \\
Merope & 200007770 & B6IVe & 4.305 & 4 & \tablenotemark{a} & SPB \\
Maia & 200007771 & B8III & 4.305 & 4 & \tablenotemark{a} & $\alpha^2$\,CVn \\
Taygeta & 200007772 & B6IV & 4.448 & 4 & \tablenotemark{a} & SPB \\
Pleione & 200007773 & B8Vne & 5.192 & 4 & \tablenotemark{a} & SPB \\
$\gamma$ Tau & 200007765 & G9.5IIIabCN0.5 & 3.474 & 4 &  & RG \\
$\delta$1 Tau & 200007766 & G9.5IIICN0.5 & 3.585 & 4 &  & RG \\
Spica & 212573842 & B1V & 0.97 & 6, 17 & Normal Mask & SPB? \\
69 Vir & 212356048 & K0III-IIIbCN1.5CH0.5 & 4.75 & 6 &  & -- \\
Ascella & 200062593 & A2.5Va & 2.585 & 7 &  & GD \\
Albaldah & 200062592 & F2II-III & 2.88 & 7 &  & GD \\
$\tau$ Sgr & 200062591 & K1.5IIIb & 3.31 & 7 &  & RG \\
$\xi$2 Sgr & 200062590 & G8/K0II/III & 3.51 & 7 &  & RG \\
$o$ Sgr & 200062589 & G9IIIb & 3.77 & 7 &  & RG \\
52 Sgr & 200062585 & B8/9V & 4.598 & 7 &  & SPB \\
Ainalrami & 200062588 & K1II & 4.845 & 7 &  & -- \\
$\psi$ Sgr & 200062584 & K0/1III+A/F & 4.85 & 7 &  & -- \\
43 Sgr & 200062587 & G8II-III & 4.878 & 7 &  & -- \\
$\nu$2 Sgr & 200062586 & K3-II-III:CN1Ba1 & 4.98 & 7 &  & RG \\
$\epsilon$ Psc & 200068392 & G9IIIbFe-2 & 4.28 & 8 &  & RG \\
Revati & 200068393 & A7IV & 5.187 & 8 &  & DS \\
80 Psc & 200068394 & F2V & 5.5 & 8 &  & GD \\
42 Cet & 200068399 & G8IV+A(8) & 5.87 & 8 &  & -- \\
33 Cet & 200068395 & K4/5III & 5.942 & 8 &  & -- \\
60 Psc & 200068396 & G8III & 5.961 & 8 &  & -- \\
73 Psc & 200068397 & K5III & 6.007 & 8 &  & -- \\
WW Psc & 200068398 & M2.5III & 6.14 & 8 &  & -- \\
\hline
\end{tabular}
\end{table*}

% \begin{table*}
\caption{Stars in Campaign 9 observed with halo photometry in K2.\label{table_1}}
\begin{tabular}{ccccccc}
\hline \hline
Name & EPIC & RA (J2000) & Dec (J2000) & Spectral & V & Campaign \\
 &  & (deg) & (deg) & Type & mag &  \\
\hline
HR 6766 & 200069361 & 272.021137 & -28.457424 & G7:IIIbCN-1CH-3.5HK+1 & 4.56 & 9 \\
HR 6842 & 200069360 & 274.513094 & -27.04213 & K3II & 4.627 & 9 \\
4 Sgr & 200069357 & 269.947601 & -23.815818 & A0 & 4.724 & 9 \\
11 Sgr & 200069358 & 272.931094 & -23.701391 & K0III & 4.98 & 9 \\
7 Sgr & 200069362 & 270.713151 & -24.282028 & F2II-III & 5.34 & 9 \\
15 Sgr & 200069359 & 273.80418 & -20.728554 & O9.7Iab & 5.37 & 9 \\
HR 6838 & 200069363 & 274.298269 & -17.37435 & K2III & 5.75 & 9 \\
Y Sgr & 200069364 & 275.34515 & -18.859826 & F8II & 5.75 & 9 \\
HR 6716 & 200069365 & 270.476773 & -22.780204 & B0Iab/b & 5.77 & 9 \\
HR 6681 & 200069366 & 269.079263 & -15.812584 & A0V & 5.929 & 9 \\
9 Sgr & 200069368 & 270.968745 & -24.361063 & O4V((f))z & 5.97 & 9 \\
16 Sgr & 200069367 & 273.803883 & -20.388154 & O9.5III & 6.02 & 9 \\
HR 6825 & 200069369 & 273.877872 & -18.661964 & ApSi & 6.15 & 9 \\
63 Oph & 200069370 & 268.725668 & -24.886798 & O8II((f)) & 6.2 & 9 \\
HR 6679 & 200069373 & 268.97931 & -18.801918 & A1V & 6.469 & 9 \\
HD 165784 & 200069371 & 272.161183 & -21.44927 & A2Iab & 6.58 & 9 \\
HD 161083 & 200069374 & 266.100216 & -22.194983 & F0V & 6.58 & 9 \\
5 Sgr & 200069372 & 270.048298 & -24.284432 & K0III & 6.64 & 9 \\
HD 167576 & 200069378 & 274.239359 & -27.716096 & K1III & 6.66 & 9 \\
HR 6773 & 200069380 & 272.225749 & -25.473139 & B3/5IV & 6.71 & 9 \\
HD 163296 & 200071159 & 269.088907 & -21.956371 & A1Vep & 6.85 & 9 \\
HD 165052 & 200069379 & 271.293504 & -24.398154 & O5.5:Vz+O8:V & 6.87 & 9 \\
17 Sgr & 200069375 & 274.147867 & -20.544369 & G8/K0III & 6.886 & 9 \\
HD 169966 & 200069376 & 277.029565 & -22.999934 & G8/K0III & 6.97 & 9 \\
HD 162030 & 200069377 & 267.489563 & -24.207101 & K1III & 7.02 & 9 \\
\hline
\end{tabular}
\end{table*}

% \begin{table*}
\caption{All stars observed with halo photometry in K2 (cont'd).}
\begin{tabular}{ccccccc}
\hline \hline
Name & EPIC & RA (J2000) & Dec (J2000) & Spectral & V & Campaign \\
 &  & (deg) & (deg) & Type & mag &  \\
\hline
81 Aqr & 200164173 & 345.348622 & -7.061254 & K4III & 6.215 & 12 \\
HR 8897 & 200164174 & 350.883513 & 0.290695 & K4III & 6.34 & 12 \\
Aldebaran & 200173843 & 68.980934 & 16.509007 & K5+III & 0.86 & 13 \\
tet2 Tau & 200173845 & 67.165927 & 15.87053 & A7III & 3.41 & 13 \\
tet2 Tau & 200173879 & 67.165992 & 15.870493 & A7III & 3.41 & 13 \\
eps Tau & 200173844 & 67.154639 & 19.179692 & G9.5IIICN0.5 & 3.53 & 13 \\
tet1 Tau & 200173846 & 67.14417 & 15.961688 & G9IIIFe-0.5 & 3.84 & 13 \\
kap1 Tau & 200173847 & 66.342857 & 22.293035 & A7IV-V & 4.201 & 13 \\
kap1 Tau & 200173880 & 66.343004 & 22.292931 & A7IV-V & 4.201 & 13 \\
del3 Tau & 200173849 & 66.372261 & 17.926961 & A2IV-Vs & 4.25 & 13 \\
tau Tau & 200173850 & 70.557694 & 22.954783 & B3V & 4.258 & 13 \\
ups Tau & 200173848 & 66.577858 & 22.812849 & A8Vn & 4.282 & 13 \\
ups Tau & 200173881 & 66.577941 & 22.812731 & A8Vn & 4.282 & 13 \\
rho Tau & 200173851 & 68.456844 & 14.858859 & A8V & 4.65 & 13 \\
rho Tau & 200173882 & 68.45714 & 14.856193 & A8V & 4.65 & 13 \\
11 Ori & 200173853 & 76.142365 & 15.403705 & A1VpSiCr & 4.661 & 13 \\
HR 1427 & 200173855 & 67.640376 & 16.193275 & A6IV & 4.764 & 13 \\
HR 1427 & 200173883 & 67.640225 & 16.193224 & A6IV & 4.764 & 13 \\
15 Ori & 200173854 & 77.42463 & 15.597631 & F2IV & 4.82 & 13 \\
75 Tau & 200173852 & 67.110364 & 16.359293 & K1IIIb & 4.969 & 13 \\
97 Tau & 200173857 & 72.84359 & 18.840322 & A7IV-V & 5.085 & 13 \\
97 Tau & 200173884 & 72.843684 & 18.84018 & A7IV-V & 5.085 & 13 \\
HR 1684 & 200173856 & 77.923187 & 16.045798 & K5III & 5.163 & 13 \\
kap2 Tau & 200173859 & 66.354939 & 22.199235 & F0Vn & 5.264 & 13 \\
kap2 Tau & 200173885 & 66.354994 & 22.199163 & F0Vn & 5.264 & 13 \\
56 Tau & 200173861 & 64.90355 & 21.772847 & A0VpSi & 5.346 & 13 \\
81 Tau & 200173860 & 67.662125 & 15.691144 & Am & 5.454 & 13 \\
53 Tau & 200173864 & 64.859035 & 21.141481 & B9Vsp & 5.482 & 13 \\
HR 1585 & 200173858 & 74.343209 & 17.152963 & K1III & 5.49 & 13 \\
80 Tau & 200173866 & 67.536514 & 15.637471 & F0V & 5.552 & 13 \\
51 Tau & 200173865 & 64.597374 & 21.578461 & F0V & 5.631 & 13 \\
HR 1403 & 200173867 & 67.004481 & 21.619624 & Am & 5.711 & 13 \\
89 Tau & 200173868 & 69.540041 & 16.032569 & F0V & 5.776 & 13 \\
HR 1576 & 200173871 & 73.959576 & 15.038117 & B9V & 5.776 & 13 \\
98 Tau & 200173870 & 74.539067 & 25.050123 & A0V & 5.785 & 13 \\
\hline
\end{tabular}
\end{table*}

% \begin{table*}
\caption{Stars in Campaign 13 observed with halo photometry in K2.\label{table_3}}
\begin{tabular}{ccccccc}
\hline \hline
Name & EPIC & Spectral & V & Campaign & Notes & Class \\
 &  &  &  &  &  &  \\
\hline
81 Aqr & 200164173 & K4III & 6.215 & 12 &  & RG \\
HR 8897 & 200164174 & K4III & 6.34 & 12 &  & -- \\
$\alpha$ Tau & 200173843 & K5+III & 0.86 & 13 & \tablenotemark{c} & -- \\
$\theta$2 Tau & 200173845 & A7III & 3.41 & 13 & SC & $\delta\,\text{Sct}$ \\
$\epsilon$ Tau & 200173844 & G9.5IIICN0.5 & 3.53 & 13 & \tablenotemark{d} & RG \\
$\theta$1 Tau & 200173846 & G9IIIFe-0.5 & 3.84 & 13 &  & -- \\
$\kappa$1 Tau & 200173847 & A7IV-V & 4.201 & 13 & SC & $\delta\,\text{Sct}$ \\
$\delta$3 Tau & 200173849 & A2IV-Vs & 4.25 & 13 & C4 & $\gamma\,\text{Dor}$ \\
$\tau$ Tau & 200173850 & B3V & 4.258 & 13 &  & SPB \\
$\upsilon$ Tau & 200173848 & A8Vn & 4.282 & 13 & SC & $\delta\,\text{Sct}$ \\
$\rho$ Tau & 200173851 & A8V & 4.65 & 13 & SC & $\delta\,\text{Sct}$ \\
11 Ori & 200173853 & A1VpSiCr & 4.661 & 13 &  & EB \\
HR 1427 & 200173855 & A6IV & 4.764 & 13 & SC & $\gamma\,\text{Dor}$ \\
15 Ori & 200173854 & F2IV & 4.82 & 13 &  & $\gamma\,\text{Dor}$ \\
75 Tau & 200173852 & K1IIIb & 4.969 & 13 &  & RG \\
97 Tau & 200173857 & A7IV-V & 5.085 & 13 & SC & $\delta\,\text{Sct}$ \\
HR 1684 & 200173856 & K5III & 5.163 & 13 &  & -- \\
$\kappa$2 Tau & 200173859 & F0Vn & 5.264 & 13 & SC & $\delta\,\text{Sct}$ \\
56 Tau & 200173861 & A0VpSi & 5.346 & 13 &  & $\gamma\,\text{Dor}$ \\
81 Tau & 200173860 & Am & 5.454 & 13 &  & $\gamma\,\text{Dor}$ \\
53 Tau & 200173864 & B9Vsp & 5.482 & 13 &  & SPB \\
HR 1585 & 200173858 & K1III & 5.49 & 13 &  & RG \\
80 Tau & 200173866 & F0V & 5.552 & 13 &  & $\gamma\,\text{Dor}$ \\
51 Tau & 200173865 & F0V & 5.631 & 13 &  & $\delta\,\text{Sct}$ \\
HR 1403 & 200173867 & Am & 5.711 & 13 &  & Hybrid \\
89 Tau & 200173868 & F0V & 5.776 & 13 &  & $\delta\,\text{Sct}$ \\
HR 1576 & 200173871 & B9V & 5.776 & 13 &  & SPB \\
98 Tau & 200173870 & A0V & 5.785 & 13 &  & $\gamma\,\text{Dor}$ \\
99 Tau & 200173862 & K0III & 5.806 & 13 &  & RG \\
105 Tau & 200173869 & B2Ve & 5.92 & 13 &  & SPB \\
HR 1554 & 200173874 & F2IVn & 5.961 & 13 &  & $\delta\,\text{Sct}$ \\
HR 1385 & 200173875 & F4V & 5.965 & 13 & C4 & $\gamma\,\text{Dor} /\delta\,\text{Sct}$ \\
HR 1741 & 200173873 & K0III & 6.107 & 13 &  & -- \\
\hline
\end{tabular}
\end{table*}

% \begin{table*}
\caption{All stars observed with halo photometry in K2 (cont'd).}
\begin{tabular}{ccccccc}
\hline \hline
Name & EPIC & RA (J2000) & Dec (J2000) & Spectral & V & Campaign \\
 &  & (deg) & (deg) & Type & mag &  \\
\hline
43 Leo & 200182930 & 155.751349 & 6.541923 & K3III & 6.08 & 14 \\
Dschubba & 200194910 & 240.0833554 & -22.62170643 & B0.3IV & 2.32 & 15 \\
Zubenelhakrabi & 200194911 & 233.8815784 & -14.78953551 & G8.5III & 3.91 & 15 \\
iot1 Lib & 200194912 & 228.0553761 & -19.7917109 & B9IVpSi & 4.54 & 15 \\
41 Lib & 200194913 & 234.7273243 & -19.30189583 & G8III/IV & 5.359 & 15 \\
zet4 Lib & 200194914 & 233.2300896 & -16.85284783 & B3V & 5.499 & 15 \\
HR 5762 & 200194915 & 233.1529208 & -19.6704581 & A2IV & 5.52 & 15 \\
HR 5806 & 200194916 & 234.4501566 & -23.1416961 & K0III & 5.79 & 15 \\
zet3 Lib & 200194917 & 232.6683426 & -16.60946629 & K0III & 5.806 & 15 \\
HR 5810 & 200194918 & 234.5678373 & -21.01632868 & K0III & 5.816 & 15 \\
iot2 Lib & 200194919 & 228.3299554 & -19.6475503 & A2V & 6.066 & 15 \\
HR 5620 & 200194920 & 226.6130965 & -22.03182838 & K0III & 6.14 & 15 \\
28 Lib & 200194921 & 230.2236529 & -18.15865908 & G8II/III & 6.17 & 15 \\
HD 138810 & 200194958 & 233.7482933 & -17.13883858 & K1(III)(+G) & 7.02 & 15 \\
Asellus Australis & 200200356 & 131.1712467 & 18.154306 & K0+IIIb & 3.94 & 16 \\
Acubens & 200200357 & 134.6217613 & 11.85770033 & kA7VmF0/2III/IVSr & 4.249 & 16 \\
ksi Cnc & 200200358 & 137.3397219 & 22.04544592 & G8.5IIIFe-0.5CH-1 & 5.149 & 16 \\
omi1 Cnc & 200200360 & 134.3122908 & 15.3227667 & A5III & 5.22 & 16 \\
eta Cnc & 200200359 & 128.1770667 & 20.44116292 & K3III & 5.325 & 16 \\
45 Cnc & 200200728 & 130.8013754 & 12.68087381 & A3III:+G7III & 5.65 & 16 \\
45 Cnc & 200200728 & 130.8013754 & 12.68087381 & A3III:+G7III & 5.65 & 16 \\
omi2 Cnc & 200200361 & 134.3966669 & 15.58128181 & F0IV & 5.677 & 16 \\
50 Cnc & 200200363 & 131.7334112 & 12.10995057 & A1Vp & 5.885 & 16 \\
Spica & 200213067 & 201.2982474 & -11.16131949 & B1V & 0.97 & 17 \\
82 Vir & 200213053 & 205.4032356 & -8.70298448 & M1+III & 5.01 & 17 \\
76 Vir & 200213054 & 203.2419673 & -10.16500253 & G8III & 5.21 & 17 \\
68 Vir & 200213055 & 201.6798633 & -12.70766332 & K5III & 5.25 & 17 \\
80 Vir & 200213056 & 203.8804021 & -5.39619162 & K0III & 5.706 & 17 \\
HR 5106 & 200213057 & 203.6685425 & -13.21432544 & A0V & 5.932 & 17 \\
HR 5059 & 200213058 & 201.5475623 & -1.19247178 & A8V & 5.965 & 17 \\
gam Cnc & 200233186 & 130.8214508 & 21.46850022 & A1IV & 4.652 & 18 \\
zet Cnc & 200233643 & 123.0530265 & 17.64776708 & F8V+G0V & 4.67 & 18 \\
eta Cnc & 200233187 & 128.1770667 & 20.44116292 & K3III & 5.325 & 18 \\
60 Cnc & 200233188 & 133.98145 & 11.62602 & K5III & 5.44 & 18 \\
49 Cnc & 200233189 & 131.1876504 & 10.08166753 & A1VpHgMnSiEu & 5.66 & 18 \\
HR 3264 & 200233190 & 125.08739 & 20.74772 & K1III & 5.798 & 18 \\
50 Cnc & 200233191 & 131.7334112 & 12.10995057 & A1Vp & 5.885 & 18 \\
29 Cnc & 200233192 & 127.1555775 & 14.21082345 & A5V & 5.948 & 18 \\
HR 3222 & 200233193 & 123.2488715 & 16.51431877 & G8III & 6.047 & 18 \\
21 Cnc & 200233196 & 125.9800391 & 10.63205666 & M2III & 6.08 & 18 \\
25 Cnc & 200233644 & 126.45782 & 17.04627 & F5IIIm? & 6.1 & 18 \\
HR 3558 & 200233195 & 134.284504 & 17.14374897 & K1III & 6.146 & 18 \\
HR 3541 & 200233194 & 133.84534 & 17.23128 & C-N4.5 & 6.4 & 18 \\
\hline
\end{tabular}
\end{table*}

% \input{all_stars_6}
\begin{table*}
\caption{All stars observed with halo photometry in K2 (cont'd).}
\caption{All stars observed with halo photometry in K2.\label{table_all}}
\begin{tabular}{ccccccc}
\hline \hline
Name & EPIC & RA (J2000) & Dec (J2000) & Spectral & V & Campaign \\
 &  & (deg) & (deg) & Type & mag &  \\
\hline
Ascella & 200062593 & 285.65184 & -29.879815 & A2.5Va & 2.585 & 7 \\
Albaldah & 200062592 & 287.441295 & -21.024023 & F2II-III & 2.88 & 7 \\
tau Sgr & 200062591 & 286.733938 & -27.671395 & K1.5IIIb & 3.31 & 7 \\
ksi2 Sgr & 200062590 & 284.432465 & -21.106731 & G8/K0II/III & 3.51 & 7 \\
omi Sgr & 200062589 & 286.17119 & -21.741407 & G9IIIb & 3.77 & 7 \\
52 Sgr & 200062585 & 294.176404 & -24.885019 & B8/9V & 4.598 & 7 \\
Ainalrami & 200062588 & 283.542904 & -22.744355 & K1II & 4.845 & 7 \\
psi Sgr & 200062584 & 288.884973 & -25.257284 & K0/1III+A/F & 4.85 & 7 \\
43 Sgr & 200062587 & 289.409117 & -18.953224 & G8II-III & 4.878 & 7 \\
nu2 Sgr & 200062586 & 283.779491 & -22.671559 & K3-II-III:CN1Ba1 & 4.98 & 7 \\
eps Psc & 200068392 & 15.736117 & 7.889231 & G9IIIbFe-2 & 4.28 & 8 \\
Revati & 200068393 & 18.43412 & 7.574624 & A7IV & 5.187 & 8 \\
80 Psc & 200068394 & 17.091325 & 5.648604 & F2V & 5.5 & 8 \\
42 Cet & 200068399 & 19.951281 & -0.509707 & G8IV+A(8) & 5.87 & 8 \\
33 Cet & 200068395 & 17.639603 & 2.445331 & K4/5III & 5.942 & 8 \\
60 Psc & 200068396 & 11.848427 & 6.740724 & G8III & 5.961 & 8 \\
73 Psc & 200068397 & 16.219136 & 5.656351 & K5III & 6.007 & 8 \\
WW Psc & 200068398 & 14.957207 & 6.483094 & M2.5III & 6.14 & 8 \\
HR 243 & 200068400 & 12.826105 & 3.38449 & G8/K0II/III & 6.368 & 8 \\
HR 161 & 200068401 & 9.377393 & 3.135111 & K3III & 6.407 & 8 \\
HR 6766 & 200069361 & 272.021137 & -28.457424 & G7:IIIbCN-1CH-3.5HK+1 & 4.56 & 9 \\
HR 6842 & 200069360 & 274.513094 & -27.04213 & K3II & 4.627 & 9 \\
4 Sgr & 200069357 & 269.947601 & -23.815818 & A0 & 4.724 & 9 \\
11 Sgr & 200069358 & 272.931094 & -23.701391 & K0III & 4.98 & 9 \\
7 Sgr & 200069362 & 270.713151 & -24.282028 & F2II-III & 5.34 & 9 \\
15 Sgr & 200069359 & 273.80418 & -20.728554 & O9.7Iab & 5.37 & 9 \\
HR 6838 & 200069363 & 274.298269 & -17.37435 & K2III & 5.75 & 9 \\
Y Sgr & 200069364 & 275.34515 & -18.859826 & F8II & 5.75 & 9 \\
HR 6716 & 200069365 & 270.476773 & -22.780204 & B0Iab/b & 5.77 & 9 \\
HR 6681 & 200069366 & 269.079263 & -15.812584 & A0V & 5.929 & 9 \\
9 Sgr & 200069368 & 270.968745 & -24.361063 & O4V((f))z & 5.97 & 9 \\
16 Sgr & 200069367 & 273.803883 & -20.388154 & O9.5III & 6.02 & 9 \\
HR 6825 & 200069369 & 273.877872 & -18.661964 & ApSi & 6.15 & 9 \\
63 Oph & 200069370 & 268.725668 & -24.886798 & O8II((f)) & 6.2 & 9 \\
HR 6679 & 200069373 & 268.97931 & -18.801918 & A1V & 6.469 & 9 \\
HD 165784 & 200069371 & 272.161183 & -21.44927 & A2Iab & 6.58 & 9 \\
HD 161083 & 200069374 & 266.100216 & -22.194983 & F0V & 6.58 & 9 \\
5 Sgr & 200069372 & 270.048298 & -24.284432 & K0III & 6.64 & 9 \\
HD 167576 & 200069378 & 274.239359 & -27.716096 & K1III & 6.66 & 9 \\
HR 6773 & 200069380 & 272.225749 & -25.473139 & B3/5IV & 6.71 & 9 \\
HD 163296 & 200071159 & 269.088907 & -21.956371 & A1Vep & 6.85 & 9 \\
HD 165052 & 200069379 & 271.293504 & -24.398154 & O5.5:Vz+O8:V & 6.87 & 9 \\
17 Sgr & 200069375 & 274.147867 & -20.544369 & G8/K0III & 6.886 & 9 \\
HD 169966 & 200069376 & 277.029565 & -22.999934 & G8/K0III & 6.97 & 9 \\
HD 162030 & 200069377 & 267.489563 & -24.207101 & K1III & 7.02 & 9 \\
Porrima & 200084004 & 190.41486 & -1.449475 & F1V+F0mF2V & 2.74 & 10 \\
Zaniah & 200084005 & 184.97638 & -0.667183 & A2IV & 3.9 & 10 \\
21 Vir & 200084006 & 188.444462 & -9.452253 & B9V & 5.48 & 10 \\
FW Vir & 200084007 & 189.593819 & 1.854722 & M3+IIICa0.5 & 5.71 & 10 \\
HR 4837 & 200084008 & 190.908208 & -1.57638 & G8III & 5.918 & 10 \\
HR 4591 & 200084009 & 180.256803 & -1.768302 & K1III & 6.316 & 10 \\
HR 4613 & 200084010 & 181.499356 & -3.131519 & G8/K0III & 6.364 & 10 \\
HD 107794 & 200084011 & 185.814177 & -4.974539 & K0III & 6.46 & 10 \\
tet Oph & 200128906 & 260.502159 & -24.999975 & OB & 3.26 & 11 \\
44 Oph & 200128907 & 261.592348 & -24.17599 & kA5hA9mF1III & 4.153 & 11 \\
45 Oph & 200128908 & 261.837707 & -29.868083 & F5III-IV & 4.269 & 11 \\
51 Oph & 200128909 & 262.85357 & -23.963494 & A0V & 4.81 & 11 \\
36 Oph & 200129035 & 258.83327 & -26.604429 & K2V+K1V & 5.03 & 11 \\
omi Oph & 200128910 & 259.502324 & -24.286539 &  & 5.2 & 11 \\
26 Oph & 200129034 & 255.039748 & -24.989128 & F3V & 5.731 & 11 \\
HR 6472 & 200128911 & 261.174968 & -21.441283 & K0III & 5.83 & 11 \\
HR 6366 & 200128913 & 257.196761 & -30.403635 & Fm dD & 5.911 & 11 \\
HR 6365 & 200128912 & 257.062511 & -17.608806 & K0III & 5.977 & 11 \\
191 Oph & 200128914 & 261.275705 & -24.243761 & K0III & 6.171 & 11 \\
kap Psc & 200164167 & 351.732716 & 1.255165 & A2VpSrCrSi & 4.94 & 12 \\
83 Aqr & 200164168 & 346.291555 & -7.693773 & F0V & 5.47 & 12 \\
24 Psc & 200164169 & 358.231585 & -3.155866 & K0II/III & 5.94 & 12 \\
HR 8759 & 200164170 & 345.382614 & -4.711516 & G5II/III & 5.933 & 12 \\
14 Psc & 200164171 & 353.53746 & -1.247154 & A2II & 5.87 & 12 \\
HR 8921 & 200164172 & 352.25226 & -9.266444 & K4/5III & 6.191 & 12 \\
81 Aqr & 200164173 & 345.348622 & -7.061254 & K4III & 6.215 & 12 \\
HR 8897 & 200164174 & 350.883513 & 0.290695 & K4III & 6.34 & 12 \\
Aldebaran & 200173843 & 68.980934 & 16.509007 & K5+III & 0.86 & 13 \\
tet2 Tau & 200173845 & 67.165927 & 15.87053 & A7III & 3.41 & 13 \\
eps Tau & 200173844 & 67.154639 & 19.179692 & G9.5IIICN0.5 & 3.53 & 13 \\
tet1 Tau & 200173846 & 67.14417 & 15.961688 & G9IIIFe-0.5 & 3.84 & 13 \\
kap1 Tau & 200173847 & 66.342857 & 22.293035 & A7IV-V & 4.201 & 13 \\
del3 Tau & 200173849 & 66.372261 & 17.926961 & A2IV-Vs & 4.25 & 13 \\
tau Tau & 200173850 & 70.557694 & 22.954783 & B3V & 4.258 & 13 \\
ups Tau & 200173848 & 66.577858 & 22.812849 & A8Vn & 4.282 & 13 \\
rho Tau & 200173851 & 68.456844 & 14.858859 & A8V & 4.65 & 13 \\
11 Ori & 200173853 & 76.142365 & 15.403705 & A1VpSiCr & 4.661 & 13 \\
HR 1427 & 200173855 & 67.640376 & 16.193275 & A6IV & 4.764 & 13 \\
15 Ori & 200173854 & 77.42463 & 15.597631 & F2IV & 4.82 & 13 \\
75 Tau & 200173852 & 67.110364 & 16.359293 & K1IIIb & 4.969 & 13 \\
97 Tau & 200173857 & 72.84359 & 18.840322 & A7IV-V & 5.085 & 13 \\
HR 1684 & 200173856 & 77.923187 & 16.045798 & K5III & 5.163 & 13 \\
kap2 Tau & 200173859 & 66.354939 & 22.199235 & F0Vn & 5.264 & 13 \\
56 Tau & 200173861 & 64.90355 & 21.772847 & A0VpSi & 5.346 & 13 \\
81 Tau & 200173860 & 67.662125 & 15.691144 & Am & 5.454 & 13 \\
53 Tau & 200173864 & 64.859035 & 21.141481 & B9Vsp & 5.482 & 13 \\
HR 1585 & 200173858 & 74.343209 & 17.152963 & K1III & 5.49 & 13 \\
80 Tau & 200173866 & 67.536514 & 15.637471 & F0V & 5.552 & 13 \\
51 Tau & 200173865 & 64.597374 & 21.578461 & F0V & 5.631 & 13 \\
HR 1403 & 200173867 & 67.004481 & 21.619624 & Am & 5.711 & 13 \\
89 Tau & 200173868 & 69.540041 & 16.032569 & F0V & 5.776 & 13 \\
HR 1576 & 200173871 & 73.959576 & 15.038117 & B9V & 5.776 & 13 \\
98 Tau & 200173870 & 74.539067 & 25.050123 & A0V & 5.785 & 13 \\
99 Tau & 200173862 & 74.45255 & 23.948656 & K0III & 5.806 & 13 \\
105 Tau & 200173869 & 76.981141 & 21.704531 & B2Ve & 5.92 & 13 \\
HR 1554 & 200173874 & 73.195975 & 27.897278 & F2IVn & 5.961 & 13 \\
HR 1385 & 200173875 & 66.238157 & 19.041326 & F4V & 5.965 & 13 \\
HR 1741 & 200173873 & 79.811052 & 20.133961 & K0III & 6.107 & 13 \\
HR 1633 & 200173872 & 76.090102 & 21.277497 & K0 & 6.188 & 13 \\
HR 1755 & 200173876 & 80.236334 & 19.814277 & K0III & 6.205 & 13 \\
rho Leo & 200182931 & 158.2027987 & 9.30658596 & B1Iab & 3.87 & 14 \\
58 Leo & 200182925 & 165.140102 & 3.617234 & K0.5IIIFe-0.5 & 4.838 & 14 \\
48 Leo & 200182926 & 158.700527 & 6.953542 & G8.5IIIFe-1 & 5.07 & 14 \\
53 Leo & 200182928 & 162.314054 & 10.545122 & A2V & 5.312 & 14 \\
65 Leo & 200182927 & 166.725448 & 1.955523 & K0III & 5.52 & 14 \\
35 Sex & 200182929 & 160.836978 & 4.747282 & K2II-III+K1II-III & 5.79 & 14 \\
43 Leo & 200182930 & 155.751349 & 6.541923 & K3III & 6.08 & 14 \\
Dschubba & 200194910 & 240.0833554 & -22.62170643 & B0.3IV & 2.32 & 15 \\
Zubenelhakrabi & 200194911 & 233.8815784 & -14.78953551 & G8.5III & 3.91 & 15 \\
iot1 Lib & 200194912 & 228.0553761 & -19.7917109 & B9IVpSi & 4.54 & 15 \\
41 Lib & 200194913 & 234.7273243 & -19.30189583 & G8III/IV & 5.359 & 15 \\
zet4 Lib & 200194914 & 233.2300896 & -16.85284783 & B3V & 5.499 & 15 \\
HR 5762 & 200194915 & 233.1529208 & -19.6704581 & A2IV & 5.52 & 15 \\
HR 5806 & 200194916 & 234.4501566 & -23.1416961 & K0III & 5.79 & 15 \\
zet3 Lib & 200194917 & 232.6683426 & -16.60946629 & K0III & 5.806 & 15 \\
HR 5810 & 200194918 & 234.5678373 & -21.01632868 & K0III & 5.816 & 15 \\
iot2 Lib & 200194919 & 228.3299554 & -19.6475503 & A2V & 6.066 & 15 \\
HR 5620 & 200194920 & 226.6130965 & -22.03182838 & K0III & 6.14 & 15 \\
28 Lib & 200194921 & 230.2236529 & -18.15865908 & G8II/III & 6.17 & 15 \\
HD 138810 & 200194958 & 233.7482933 & -17.13883858 & K1(III)(+G) & 7.02 & 15 \\
Asellus Australis & 200200356 & 131.1712467 & 18.154306 & K0+IIIb & 3.94 & 16 \\
Acubens & 200200357 & 134.6217613 & 11.85770033 & kA7VmF0/2III/IVSr & 4.249 & 16 \\
ksi Cnc & 200200358 & 137.3397219 & 22.04544592 & G8.5IIIFe-0.5CH-1 & 5.149 & 16 \\
omi1 Cnc & 200200360 & 134.3122908 & 15.3227667 & A5III & 5.22 & 16 \\
eta Cnc & 200200359 & 128.1770667 & 20.44116292 & K3III & 5.325 & 16, 18 \\
45 Cnc & 200200728 & 130.8013754 & 12.68087381 & A3III:+G7III & 5.65 & 16 \\
omi2 Cnc & 200200361 & 134.3966669 & 15.58128181 & F0IV & 5.677 & 16 \\
50 Cnc & 200200363 & 131.7334112 & 12.10995057 & A1Vp & 5.885 & 16, 18 \\
Spica & 200213067 & 201.2982474 & -11.16131949 & B1V & 0.97 & 17 \\
82 Vir & 200213053 & 205.4032356 & -8.70298448 & M1+III & 5.01 & 17 \\
76 Vir & 200213054 & 203.2419673 & -10.16500253 & G8III & 5.21 & 17 \\
68 Vir & 200213055 & 201.6798633 & -12.70766332 & K5III & 5.25 & 17 \\
80 Vir & 200213056 & 203.8804021 & -5.39619162 & K0III & 5.706 & 17 \\
HR 5106 & 200213057 & 203.6685425 & -13.21432544 & A0V & 5.932 & 17 \\
HR 5059 & 200213058 & 201.5475623 & -1.19247178 & A8V & 5.965 & 17 \\
gam Cnc & 200233186 & 130.8214508 & 21.46850022 & A1IV & 4.652 & 18 \\
zet Cnc & 200233643 & 123.0530265 & 17.64776708 & F8V+G0V & 4.67 & 18 \\
60 Cnc & 200233188 & 133.98145 & 11.62602 & K5III & 5.44 & 18 \\
49 Cnc & 200233189 & 131.1876504 & 10.08166753 & A1VpHgMnSiEu & 5.66 & 18 \\
HR 3264 & 200233190 & 125.08739 & 20.74772 & K1III & 5.798 & 18 \\
29 Cnc & 200233192 & 127.1555775 & 14.21082345 & A5V & 5.948 & 18 \\
HR 3222 & 200233193 & 123.2488715 & 16.51431877 & G8III & 6.047 & 18 \\
21 Cnc & 200233196 & 125.9800391 & 10.63205666 & M2III & 6.08 & 18 \\
25 Cnc & 200233644 & 126.45782 & 17.04627 & F5IIIm? & 6.1 & 18 \\
HR 3558 & 200233195 & 134.284504 & 17.14374897 & K1III & 6.146 & 18 \\
HR 3541 & 200233194 & 133.84534 & 17.23128 & C-N4.5 & 6.4 & 18 \\
\hline
\end{tabular}
\end{table*}

\begin{deluxetable}{cccc}
\tablecaption{Global asteroseismic parameters for the 31 red giants for which solar-like oscillations were detected.\label{rgs}}
\tablehead{\colhead{Name} & \colhead{EPIC} & \colhead{\numax} & \colhead{\Dnu}\\ \colhead{} & \colhead{} & \colhead{(\muHz)} & \colhead{(\muHz)}}
\startdata
$\gamma$ Tau & 200007765 & 62.9 $\pm$ 1.44 & 5.6 $\pm$ 0.17 \\
$\delta^{1}$ Tau & 200007766 & 62.6 $\pm$ 1.74 & 5.7 $\pm$ 0.07 \\
$\nu^{2}$ Sgr & 200062586 & 7.3 $\pm$ 0.15 & 1.3 $\pm$ 0.05 \\
$o$ Sgr & 200062589 & 46.3 $\pm$ 1.02 & 4.8 $\pm$ 0.06 \\
$\xi^{2}$ Sgr & 200062590 & 11.7 $\pm$ 0.65 & 1.9 $\pm$ 0.15 \\
$\tau$ Sgr & 200062591 & 19.8 $\pm$ 0.80 & 2.5 $\pm$ 0.07 \\
$\pi$ Sgr & 200062592 & 47.0 $\pm$ 0.43 & 6.0 $\pm$ 0.20 \\
$\epsilon$ Psc & 200068392 & 33.3 $\pm$ 1.22 & 3.6 $\pm$ 0.07 \\
11 Sgr & 200069358 & 38.0 $\pm$ 0.84 & 4.0 $\pm$ 0.13 \\
HR 6766 & 200069361 & 20.6 $\pm$ 4.19 & 2.4 $\pm$ 0.41 \\
7 Sgr & 200069362 & 13.6 $\pm$ 0.97 & 2.0 $\pm$ 0.20 \\
HR 6716 & 200069365 & 10.7 $\pm$ 3.38 & 1.8 $\pm$ 0.28 \\
16 Sgr & 200069367 & 13.8 $\pm$ 0.34 & 2.2 $\pm$ 0.11 \\
5 Sgr & 200069372 & 47.8 $\pm$ 0.95 & 4.6 $\pm$ 0.05 \\
191 Oph & 200128914 & 29.2 $\pm$ 0.92 & 3.9 $\pm$ 0.10 \\
HR 8759 & 200164170 & 10.1 $\pm$ 0.39 & 1.6 $\pm$ 0.05 \\
81 Aqr & 200164173 & 11.4 $\pm$ 0.23 & 1.7 $\pm$ 0.06 \\
$\epsilon$ Tau & 200173844 & 54.5 $\pm$ 1.44 & 5.1 $\pm$ 0.13 \\
75 Tau & 200173852 & 35.0 $\pm$ 0.96 & 4.2 $\pm$ 0.04 \\
HR 1585 & 200173858 & 9.4 $\pm$ 1.01 & 1.5 $\pm$ 0.10 \\
99 Tau & 200173862 & 21.4 $\pm$ 1.07 & 2.4 $\pm$ 0.07 \\
HR 1755 & 200173876 & 18.8 $\pm$ 0.41 & 2.0 $\pm$ 0.04 \\
58 Leo & 200182925 & 17.0 $\pm$ 0.46 & 2.0 $\pm$ 0.23 \\
48 Leo & 200182926 & 53.3 $\pm$ 0.79 & 5.4 $\pm$ 0.04 \\
65 Leo & 200182927 & 61.6 $\pm$ 1.38 & 6.4 $\pm$ 0.03 \\
35 Sex & 200182929 & 11.5 $\pm$ 0.15 & 1.5 $\pm$ 0.05 \\
43 Leo & 200182930 & 71.6 $\pm$ 2.81 & 7.2 $\pm$ 0.08 \\
$\gamma$ Lib & 200194911 & 34.9 $\pm$ 0.98 & 3.6 $\pm$ 0.10 \\
41 Lib & 200194913 & 54.3 $\pm$ 1.79 & 5.2 $\pm$ 0.03 \\
HR 5806 & 200194916 & 53.2 $\pm$ 0.75 & 4.9 $\pm$ 0.06 \\
$\zeta^{3}$ Lib & 200194917 & 44.2 $\pm$ 1.00 & 3.6 $\pm$ 0.26 \\
HR 5810 & 200194918 & 45.0 $\pm$ 0.46 & 4.5 $\pm$ 0.03 \\
HR 5620 & 200194920 & 96.8 $\pm$ 0.74 & 9.3 $\pm$ 0.03 \\
28 Lib & 200194921 & 41.0 $\pm$ 0.86 & 4.1 $\pm$ 0.17 \\
$\eta$ Cnc & 200200359 & 22.9 $\pm$ 0.86 & 2.7 $\pm$ 0.03 \\
76 Vir & 200213054 & 40.0 $\pm$ 2.62 & 3.8 $\pm$ 0.09 \\
80 Vir & 200213056 & 37.0 $\pm$ 1.83 & 4.4 $\pm$ 0.08 \\
HR 3264 & 200233190 & 22.9 $\pm$ 0.17 & 3.0 $\pm$ 0.18
\enddata
\end{deluxetable}

%% This command is needed to show the entire author+affilation list when
%% the collaboration and author truncation commands are used.  It has to
%% go at the end of the manuscript.
%\allauthors

%% Include this line if you are using the \added, \replaced, \deleted
%% commands to see a summary list of all changes at the end of the article.
%\listofchanges

\end{document}

% End of file `sample62.tex'.
